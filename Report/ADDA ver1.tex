\documentclass[cs4size,a4paper]{ctexart}   
%==================== 数学符号公式 ============
\usepackage{amsmath}                 % AMS LaTeX宏包
\usepackage[style=1]{mdframed}
\usepackage{amsthm}
\usepackage{amsfonts}
\usepackage{mathrsfs}                % 英文花体字体
\usepackage{bm}                      % 数学公式中的黑斜体
\usepackage{bbding,manfnt}           % 一些图标,如 \dbend
\usepackage{lettrine}                % 首字下沉,命令\lettrine
\def\attention{\lettrine[lines=2,lraise=0,nindent=0em]{\large\textdbend\hspace{1mm}}{}}
\usepackage{longtable}
\usepackage[toc,page]{appendix}
\usepackage{geometry}                % 页边距调整
\usepackage{makecell}                % 表格内换行
\geometry{top=3.0cm,bottom=2.7cm,left=2.5cm,right=2.5cm}
%====================公式按章编号==========================
\numberwithin{equation}{section}
\numberwithin{table}{section}
\numberwithin{figure}{section}
%================= 基本格式预置 ===========================
\usepackage{fancyhdr}
\pagestyle{fancy}
\fancyhf{}  
\fancyhead[C]{\zihao{5}  \kaishu fancyhead千万别忘改了}
\fancyfoot[C]{~\zihao{5} \thepage~}
\renewcommand{\headrulewidth}{0.65pt} 
% \CTEXsetup[format={\centering\bfseries\zihao{-2}},name={第, 章}]{section}
\CTEXsetup[format={\centering\bfseries\zihao{4}}]{section}
\CTEXsetup[format={\bfseries \zihao{4}}]{subsection}
\CTEXsetup[format={\bfseries \zihao{-4}}]{subsubsection}
%================== 图形支持宏包 =========================
\usepackage{subfig}
\usepackage{graphicx}                % 嵌入png图像
\usepackage{color,xcolor}            % 支持彩色文本、底色、文本框等
\usepackage{hyperref}                % 交叉引用
\usepackage{caption}
\captionsetup{figurewithin=section}
%==================== 源码和流程图 =====================
\usepackage{listings}                % 粘贴源代码
\usepackage{xcolor}
\usepackage{color}
\definecolor{dkgreen}{rgb}{0,0.6,0}
\definecolor{gray}{rgb}{0.5,0.5,0.5}
\definecolor{mauve}{rgb}{0.58,0,0.82}
 \usepackage{xcolor}

% \renewcommand{\lstlistingname}{代码}    % 代码标题样式修改无效?!
%  \lstset{
%     basicstyle          =   \sffamily,          % 基本代码风格
%     keywordstyle        =   \bfseries,          % 关键字风格
%     commentstyle        =   \rmfamily\itshape,  % 注释的风格,斜体
%     stringstyle         =   \ttfamily,  % 字符串风格
%     flexiblecolumns,                % 别问为什么,加上这个
%     numbers             =   left,   % 行号的位置在左边
%     showspaces          =   false,  % 是否显示空格,显示了有点乱,所以不现实了
%     numberstyle         =   \zihao{5}\ttfamily,    % 行号的样式,小五号,tt等宽字体
%     showstringspaces    =   false,
%     captionpos          =   t,      % 这段代码的名字所呈现的位置,t指的是top上面
%     frame               =   lrtb,   % 显示边框
% }

%  \lstdefinestyle{SPICE}{
%     language        =   matlab, % 语言选Python
%     basicstyle      =   \zihao{-5}\ttfamily,
%     numberstyle     =   \zihao{-5}\ttfamily,
%     keywordstyle    =   \color{blue},
%     keywordstyle    =   [2] \color{teal},
%     stringstyle     =   \color{magenta},
%     commentstyle    =   \color{red}\ttfamily,
%     breaklines      =   true,   % 自动换行,建议不要写太长的行
%     columns         =   fixed,  % 如果不加这一句,字间距就不固定,很丑,必须加
%     basewidth       =   0.5em,
% }

%--------------------
\hypersetup{hidelinks}
\usepackage{booktabs}  
\usepackage{shorttoc}
\usepackage{tabu,tikz}
\usepackage{float}

\usepackage{multirow}



\tabcolsep=1ex
\tabulinesep=\tabcolsep
\newlength\tikzboxwidth
\newlength\tikzboxheight
\newcommand\tikzbox[1]{%
        \settowidth\tikzboxwidth{#1}%
        \settoheight\tikzboxheight{#1}%
        \begin{tikzpicture}
        \path[use as bounding box]
                (-0.5\tikzboxwidth,-0.5\tikzboxheight)rectangle
                (0.5\tikzboxwidth,0.5\tikzboxheight);
        \node[inner sep=\tabcolsep+0.5\arrayrulewidth,line width=0.5mm,draw=black]
                at(0,0){#1};
        \end{tikzpicture}%
        }

\makeatletter
\def\hlinew#1{%
  \noalign{\ifnum0=`}\fi\hrule \@height #1 \futurelet
   \reserved@a\@xhline}
   
\newcommand{\tabincell}[2]{\begin{tabular}{@{}#1@{}}#2\end{tabular}}%

\usepackage{subfig}

\usepackage{CJK}
\usepackage{ifthen}


\usepackage{graphicx} 
\newcommand{\HRule}{\rule{\linewidth}{0.5mm}}

\newtheorem{Theorem}{定理}
\newtheorem{Lemma}{引理} 
%%使得公式随章节自动编号
\makeatletter
\@addtoreset{equation}{section}
\makeatother
\renewcommand{\theequation}{\arabic{section}.\arabic{equation}}

%-------------------------
	
\usepackage{pythonhighlight}
\usepackage{tikz}                    
\usepackage{tikz-3dplot}
\usetikzlibrary{shapes,arrows,positioning}
%===================   正文开始    ===================
\begin{document}
\bibliographystyle{gbt7714-2005}     %论文引用格式
%===================  定理类环境定义 ===================
\newtheorem{example}{例}              % 整体编号
\newtheorem{algorithm}{算法}
\newtheorem{theorem}{定理}            % 按 section 编号
\newtheorem{definition}{定义}
\newtheorem{axiom}{公理}
\newtheorem{property}{性质}
\newtheorem{proposition}{命题}
\newtheorem{lemma}{引理}
\newtheorem{corollary}{推论}
\newtheorem{remark}{注解}
\newtheorem{condition}{条件}
\newtheorem{conclusion}{结论}
\newtheorem{assumption}{假设}
%==================重定义 ===================
\renewcommand{\contentsname}{目录}     
\renewcommand{\abstractname}{摘要} 
\renewcommand{\refname}{参考文献}     
\renewcommand{\indexname}{索引}
\renewcommand{\figurename}{图}
\renewcommand{\tablename}{表}
\renewcommand{\appendixname}{附录}
\renewcommand{\proofname}{证明}
\renewcommand{\algorithm}{算法} 
%============== 封皮和前言 =================
% \input{body/cover}
\pagestyle{plain}
\pagenumbering{arabic}
% \include{body/abstract}
\pagestyle{empty}
% \tableofcontents 
% \thispagestyle{empty}
%============== 论文正文   =================
% \pagestyle{fancy}
\pagestyle{plain}
% \include{body/chapter1}      %
% \include{body/chapter2}
% \include{body/chapter3}
% \include{body/chapter4}
% \include{body/chapter5}
% \include{body/chapter6}

\begin{center}
    \textsc{\zihao{-3} \bfseries ADDA课程项目}\\[0.3cm]
\end{center}

\begin{center}
	组员:冯俊杰、贾梓越、陈江华、陈畅
\end{center}

\section{设计目标}
\noindent
研究内容:

通过 matlab 建模或者 cadence(MOS 管级/verilogA 级,工艺不限)构建一个时钟交
织 Pipelined-SAR ADC,pipelined 级数不限,时钟校准通道数≥4 条,精度≥12 位,
TI-ADC 总采样率≥2GS/s,自己手动添加以下误差和失配,例如采样时钟 jitter 范围
100fs~5ps,timing skew 范围 200fs~5ps, inter-stage gain error± 10%等(以上参
数仅供参考), 通过静态分析方法和动态分析方法分析:TI mismatch (gain mismatch,
offset mismatch and timing skew), pipelined-SAR ADC 级间增益误差(inter-stage 
gain error), 第一级 SAR ADC 电容失配(capacitor-DAC mismatch),比较器失调,
时钟 jitter 等的影响。选择一种 time skew 校准方法进行 timing skew 误差检测并分
析。建模、仿真验证、撰写完整分析报告,在 cadence 里面完成仿真验证的有额外 bonus
分数。

\noindent
分析方法:

静态特性分析:转移特性曲线/统计方法;动态特性分析:FFT

\section{电路结构及原理}

    \subsection{Pipeline SAR ADC建模原理}
        基本架构:如图\ref{fig:circuit1}所示,两级SAR级间插入一个余量放大器。
        	\begin{figure}[H]
        		\centering
        		\includegraphics[width=0.5\textwidth]{figure/circuit1.jpg}
        		\caption{pipeline级联原理图} \label{fig:circuit1}
        	\end{figure}
        	
        总位数为
            \begin{equation}
                N=N_1+N_2+...-R
            \end{equation}
            
        N为总位数,其中R为冗余位。
        
        \textbf{\textit{在设计时需要注意以下几点:}}
        \begin{itemize}
			\item \textbf{时序设计基本原则:}$fs=1/Ts$,其中$Ts = min\{Ts_1,Ts_2,...\}$,由于采样率取决于最慢的周期,设计时尽量让每级周期近似相等,避免时序浪费。
			\item \textbf{余量传输:}如图\ref{fig:circuit2}所示,正常需要将数字量经过DAC转回模拟域,但考虑SAR采用电容开关阵列,可以加一位电容,利用SAR本身开关阵列实现DAC,得到余差。即纯SAR ADC的开关阵列正常需要N-2位电容,但若要利用电容开关阵列实现DAC功能,应当考虑N-1位电容,这就造成了前后级的结构不同,因为最后一级只需实现纯SAR ADC,无需再转换回模拟域。
				\begin{figure}[H]
					\centering
					\includegraphics[width=0.7\textwidth]{figure/circuit2.jpg}
					\caption{余量放大传输原理图} \label{fig:circuit2}
				\end{figure}
			以2级流水线为例,若第一级M1位,第二级M2位,则实际第一级的ADC需要M1-1位电容,而第二即只需要M2-2位电容。
			\item \textbf{放大倍数:}
				\begin{equation}
					G=2^{M1-R-K}=2^{M1-R}\cdot\frac{Vref2}{Vref1}
				\end{equation}
			M1为前级ADC的位数。
			
			R为冗余位的位数——引入冗余的目的是为了采用冗余位校正算法来校正比较器失调和DAC建立误差,同时也能降低系统对比较器噪声的要求。
			
			k为两级间的缩减系数——例如第一级量化范围为$[0,Vref]$,第二级量化范围就可以表示为$[0,2^{-k} \cdot Vref]$,可以转化为两级参考电压之比。
			
			\item \textbf{冗余:}希望超出量化范围的信息也可以被后级SAR ADC量化。
			
			具体实现:余量放大器倍数减半并保持第二级SAR ADC量化范围不变,就可以实现1位冗余校正。
			
			模拟域和数字域对应:前级最高位和次级首位对齐,并减去失调。
			
			\item \textbf{运放误差:}运放非理想,实际情况需要考虑以下几方面的误差
				\begin{itemize}
					\item 有限增益误差
					\item 有限带宽误差
					\item 噪声和失调
				\end{itemize}
			
        \end{itemize}


    \subsection{Time-Interleaved ADC 建模原理}
		基本架构:如图\ref{fig:circuit3}所示,多路时间交织可以将等效采样率提高到但通道的m倍。
		\begin{figure}[H]
			\centering
			\includegraphics[width=1\textwidth]{figure/circuit3.png}
			\caption{时间交织原理及时序图} \label{fig:circuit3}
		\end{figure}
		\noindent
		\textbf{\textit{同样,注意建模设计要点:不同的通道间误差及其各自与频谱的对应关系}}
		在此,我们会首先回顾课上以两通道为例的失配,再将其推广到多通道,并研究不同通道间误差与频谱各自的对应关系。

		\subsubsection{失调失配}
		\textbf{以2通道为例,输入为 $x(t) = \cos(\omega t + \phi)$}
		\begin{itemize}
			\item \textbf{失调失配}:$x(t) = \cos(\omega t + \phi) + OS$
			\begin{itemize}
				\item 存在\textcolor{red}{失调}失配,$\Delta OS = OS_A - OS_B \neq 0$
				\item 量化结果:
				\[
				y[n] = \cos(\omega nT + \phi) + OS + \frac{\Delta OS}{2} \cos\left(\frac{\omega_s}{2} nT \right)
				\]
			\end{itemize}
		\end{itemize}
		\underline{\textbf{接下来,推广到多通道失调失配:}}
		\begin{equation}
			\begin{gathered}
				x_s(t)=x(t)+O(t)\\
				 \quad f_{I L}=k f_s / M \quad \text { where } k=0,1,2,3 \ldots, M-1 \\
				P_N=\frac{1}{M} \sum_{i=0}^{M-1}\left|O_i\right|^2
			\end{gathered}
		\end{equation}
		
		
		
		\subsubsection{增益失配}
		\textbf{增益失配}:$x(t) = G \cdot \cos(\omega t + \phi)$
		\begin{itemize}
			\item 存在\textcolor{red}{增益}失配,$\Delta G = G_A - G_B \neq 0$
			\item 量化结果:
			\[
			y[n] = G \cos(\omega nT + \phi) + \frac{\Delta G}{2} \cos\left[ \left(\omega - \frac{\omega_s}{2} \right) nT + \phi \right]
			\]
		\end{itemize}
		\underline{\textbf{接下来,推广到多通道增益失配:}}
		\begin{equation}
			\begin{gathered}
				x_s(t)=x(t) \times G(t) \\
				 f_{\text {Gain }} =k f_s / M \\
				f_{\text {IL }}=f_{\text {Gain }} \pm f_{\text {in }}= \pm f_{\text {in }}+\frac{k}{M} f_s \\
				 P_{\text {Total }}  =\frac{1}{2 M} \sum_{i=0}^{M-1}\left|G_i\right|^2
			\end{gathered}
		\end{equation}
		
		\subsubsection{时钟偏差}
		\textbf{采样时刻偏差}:$x(t) = \cos(\omega t + \phi + \Delta T)$
		\begin{itemize}
			\item 存在\textcolor{red}{采样时刻偏差},$\Delta T$
			\item 量化结果:
			\[
			y[n] = \cos\left(\frac{\omega \Delta T}{2}\right) \cos(\omega nT + \phi) + \sin\left(\frac{\omega \Delta T}{2}\right) \sin\left[\left(\omega - \frac{\omega_s}{2} \right) nT + \phi \right]
			\]
		\end{itemize}
		\underline{\textbf{接下来,推广到多通道采样时刻偏差:}}
		\begin{equation}
			\begin{array}{lr}
				x_s(t)=x(t-\delta t) & x_s(t)=\sin \left(2 \pi f_{i n} t\right) \cos \left(2 \pi f_{i n} \delta t\right)-\cos \left(2 \pi f_{i n} t\right) \sin \left(2 \pi f_{i n} \delta t\right) \\
				x(t)=\sin \left(2 \pi f_{i n} t\right) & x_s(t) \cong \sin \left(2 \pi f_{i n} t\right)-\left(2 \pi f_{i n} \delta t\right) \cos \left(2 \pi f_{i n} t\right) \\
				x_s(t)=\sin \left(2 \pi f_{i n} t-2 \pi f_{i n} \delta t \right) & f_{I L}= \pm f_{i n}+f_{\delta t}= \pm f_{i n}+\frac{k}{M} f_s \nonumber
			\end{array}
		\end{equation}
		
		\begin{equation}
			P_N=\frac{\left(2 \pi f_{i n}\right)^2}{2 M} \sum_{i=0}^{M-1}\left|\delta t_i\right|^2
		\end{equation}
		
		
		\subsubsection{带宽失配(但这个在本模型中未体现)}
		\textbf{带宽失配}:$x(t) = H(\omega) \cdot \cos(\omega t + \phi)$
		\begin{itemize}
			\item 存在\textcolor{red}{带宽}失配,$\Delta H(\omega) = H_A(\omega) - H_B(\omega) \neq 0$
			\item 量化结果:
			\[
			y[n] = H(\omega) \cos(\omega nT + \phi) + \frac{\Delta H(\omega)}{2} \cos\left[\left( \omega - \frac{\omega_s}{2} \right) nT + \phi \right]
			\]
		\end{itemize}
		\underline{\textbf{带宽失配未在本模型中加以考虑,只是在此做一个简单的说明}}
		
		\subsection{时间交织失配误差总结}
\begin{flushleft}
			\begin{tabular}{|l|l|l|}
			\hline \textbf{Type of mismatch} & \textbf{Effect on input} & \textbf{Spur location} \\
			\hline Offset mismatch & Additive effect & $f_{\text {IL }}=\frac{k}{M} f_s$ \\
			\hline Gain mismatch & Amplitude modulation & $f_{\text {IL }}= \pm f_{\text {in }}+\frac{k}{M} f_s$ \\
			\hline Timing mismatch & Phase modulation & $f_{\text {IL }}= \pm f_{\text {in }}+\frac{k}{M} f_s$ \\
			\hline Bandwidth mismatch & Freq.-dependent amplitude and phase modulation & $f_{I L}= \pm f_{\text {in }}+\frac{k}{M} f_s$ \\
			\hline
		\end{tabular}
\end{flushleft}
		
		



		
%============= 参考文献 =====================
% \addcontentsline{toc}{section}{参考文献}
% \bibliography{bibfile}
\clearpage
%=============  致谢  ======================
% \include{body/acknowledge}
%\include{body/appendices}

\end{document}
%%%%%%%%%% 结束 %%%%%%%%%%